\textsc{Background}.
The effective reproductive number $\Re(t)$
is critical measure of epidemic potential.
$\Re(t)$ can be calculated in near real time
using an incidence time series and the generation time distribution---%
the time between infection events in an infector-infectee pair.
In calculating $\Re(t)$, the generation time distribution
is often approximated by the serial interval distribution---%
the time between symptom onset in an infector-infectee pair.
However, such an approximation may lead to
negative values of the serial interval when transmission can 
occur prior to symptoms, such as in \covid.
\textsc{Methods}.
We developed a method to infer the generation time distribution
from parametric definitions of, and using empirical data on,
the serial interval and incubation period distributions.
We then compared estimates of $\Re(t)$ for \covid in
the Greater Toronto Area of Canada using:\
negative-permitting versus non-negative serial interval distributions,
versus the inferred generation time distribution.
\textsc{Results}.
We estimated the generation time of \covid to be
Gamma-distributed with mean $3.99$ and standard deviation $2.96$ days.
Relative to the generation time distribution,
non-negative serial interval distribution caused overestimation of $\Re(t)$
due to larger mean, while
negative-permitting serial interval distribution caused underestimation of $\Re(t)$
due to larger variance.
\textsc{Implications}.
Approximation of the generation time distribution of \covid
with non-negative or negative-permitting serial interval distributions
when calculating $\Re(t)$ may result in over or underestimation of
transmission potential, respectively.
%SM: nice abstract!