\textsc{Background}.
The effective reproductive number $\Re(t)$
is critical measure of epidemic potential.
$\Re(t)$ can be calculated in near real time
using an incidence time series and the generation time distribution---%
the time between infection events in an infector-infectee pair.
In calculating $\Re(t)$, the generation time distribution
is often approximated by the serial interval distribution---%
the time between symptom onset in an infector-infectee pair.
However, while generation time must be positive by definition,
serial interval can be negative if transmission can occur before symptoms,
such as in \covid, rendering such an approximation improper in some contexts. %SM: i like this
% JK: I think it's not the approximation that leads to negative values,
%     More that the negative values in S.I. are invalid for a G.T. distribution.
%     Also, I wanted to end background with a "problem statement",
%     hence "rendering such an approximation improper in some contexts",
%     but if there's a better way to say that I'm open!
\textsc{Methods}.
We developed a method to infer the generation time distribution
from parametric definitions of
% JK: can you clarify what empirical data means in this case?  % SM: used data.. i.e. used data on serial interval and incubation period...it does not have to be raw/person-level data. but I see what you mean, and I think leave as you have it [i.e. parametric definitions].
%     If line-listed / person-level data, our method doesn't really use that,
%     so I would kind of prefer to stick with just parametric definitions.
the serial interval and incubation period distributions.
We then compared estimates of $\Re(t)$ for \covid in
the Greater Toronto Area of Canada using:\
negative-permitting versus non-negative serial interval distributions,
versus the inferred generation time distribution.
\textsc{Results}.
We estimated the generation time of \covid to be
Gamma-distributed with mean $3.99$ and standard deviation $2.96$ days.
Relative to the generation time distribution,
non-negative serial interval distribution caused overestimation of $\Re(t)$
due to larger mean, while
negative-permitting serial interval distribution caused underestimation of $\Re(t)$
due to larger variance.
\textsc{Implications}.
Approximation of the generation time distribution of \covid
with non-negative or negative-permitting serial interval distributions
when calculating $\Re(t)$ may result in over or underestimation of
transmission potential, respectively.