The effective reproduction number $\Re(t)$ provides
an instantaneous measure of potential for epidemic growth,
after considering herd immunity and interventions.
\textcite{Cori2013} provide a method to estimate $\Re(t)$
in quasi-real time based on only two inputs:\
an incidence time series,%
\footnote{Incidence may be further stratified by
  imported versus locally generated cases
  to quantify local transmission dynamics.}
and the generation time distribution.
The generation time is defined as
the time between infection events in an infector-infectee pair.
\par
When estimating $\Re(t)$
in previous epidemics \cite{Cori2013,Ali2013,Aylward2014},
and in \covid \cite{Pan2020,Cowling2020,Leung2020,Liu2020},
the generation time has been approximated by the serial interval.  %great material and writing - clear up to this point :)
The serial interval is defined as the time between symptom onset in an infector-infectee pair,
since symptom onset is directly observable.
This approximation is reasonable for infectious diseases
where onset of infectiousness and symptoms is simultaneous \cite{Cori2013} such as in the case of SARS and Ebola . %SM, cite the use of serial interval for SARS and Ebola models (e.g. renewal equation) - https://www.ncbi.nlm.nih.gov/pmc/articles/PMC5871640/
However, potential pre-symptomatic transmission of \covid
\cite{Kimball2020,Du2020}
renders approximation of generation time by serial interval problematic.
Namely, while generation time is strictly positive,
the serial interval can be negative due to
variability in the incubation period,
such as in 59 of 468 (12.6\%) reported cases in~\cite{Du2020}.
As a result, there have been differences in     %SM:  clarify the sentence (could not follow / understand the sentence). use simpler words :).
whether or not non-negative distributions are fit
to \covid serial interval data
\cite{Du2020,Zhang2020,Nishiura2020,Ganyani2020}.
\par

%SM: I do not understand the objectives as currently worded - can we simplify the writing? the objective is not clear to me on reading this.
We aimed to show that it might be unnecessary to
fit non-negative distributions to serial interval data
for the purpose of calculating $\Re(t)$,
since a non-negative generation time distribution can be inferred
based on the incubation period and serial interval distributions.
We also explored the implications for the estimated $\Re(t)$
of using non-negative serial interval distributions
versus an inferred generation time distribution.
