The effective reproduction number $\Re(t)$ provides
an instantaneous measure of transmission potential
or the rate of spread on an epidemic.
\textcite{Cori2013} provide a method to estimate $\Re(t)$
in quasi-real time based on only two inputs:\
an incidence time series,%
\footnote{Incidence may be further stratified by
  imported versus locally generated cases
  to quantify local transmission dynamics.}
and the generation time distribution.
The generation time is defined as
the time between infection events in an infector-infectee pair.
\par
When estimating $\Re(t)$
in previous epidemics \cite{Cori2013,Ali2013,Aylward2014}
and in \covid \cite{Pan2020,Cowling2020,Leung2020,Liu2020},
the generation time has been approximated by the serial interval.
The serial interval is defined as
the time between symptom onset in an infector-infectee pair.
Unlike infection events, symptom onset is directly observable.
This approximation is reasonable for infectious diseases
where onset of infectiousness and symptoms
is effectively simultaneous \cite{Cori2013}
such for SARS and Ebola \cite{Zeng2009,Osterholm2015}.
However, potential pre-symptomatic transmission of \covid
\cite{Kimball2020,Du2020}
renders approximation of generation time by serial interval problematic.
Namely, while generation time is strictly positive,
the serial interval can be negative,
such as in 59 of 468 (12.6\%) reported cases in~\cite{Du2020},
due to variability in the incubation period.
As a result, \covid serial interval data have been used to fit both
non-negative and negative-permitting distributions,
yielding different estimates of $\Re(t)$
\cite{Du2020,Zhang2020,Ganyani2020}.
\par
We compared estimates of $\Re(t)$ and their implications using
each of the following approximations of the generation time distribution:
(a) negative-permitting distribution fit to serial interval data;
(b) non-negative distribution fit to serial interval data; and
(c) inferred generation time distribution based on
the incubation period and the serial interval distributions.
We used parametric distributions described in the literature for \covid,
and reported cases from the Greater Toronto Area (\gta), Canada.