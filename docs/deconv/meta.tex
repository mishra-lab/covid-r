\title{\vskip-8ex%
  Estimating effective reproduction number using
  generation time versus serial interval,\\
  with application to \covid\\
  in the Greater Toronto Area, Canada
}
\usepackage{authblk}
\author[1,2]{Jesse Knight}
\author[1,2,3,4]{Sharmistha Mishra}
\affil[1]{Institute of Medical Science, University of Toronto}
\affil[2]{MAP Centre for Urban Health Solutions, Unity Health Toronto}
\affil[3]{Division of Infectious Disease, Department of Medicine, University of Toronto}
\affil[4]{Institute of Health Policy, Management and Evaluation, Dalla Lana School of Public Health, University of Toronto}
\renewcommand\Affilfont{\footnotesize}
\date{
  {\textbf{Correction}}\\
  March 23, 2021\\[1ex]
  \parbox{\linewidth}{\footnotesize
  In a previous version of this work,
  the parameters $\theta = [\alpha, \beta]$ were calculated incorrectly
  because the Kullback-Leibler divergence was defined in the wrong direction.
  The impact on generation time statistics is as follows (original $\rightarrow$ fixed):
  shape ($\alpha$): $1.813 \rightarrow 1.633$,
  scale ($\beta$): $2.199 \rightarrow 2.498$,
  mean: $3.99 \rightarrow 4.08$,
  \sd: $2.96 \rightarrow 3.19$.
  The qualitative interpretation of results is unchanged,
  and the corrected numerical results are given in this version.
  We sincerely apologize for this error.\\
  The error correction is shown here:
  \scriptsize\hreftt{https://github.com/mishra-lab/covid-r/commit/aaef512}}
}