Figure~\ref{fig:deconv} shows the
serial interval and incubation period distributions
from \cite{Du2020} and \cite{Lauer2020} respectively,
and the generation time distribution estimated via the proposed method.
The \mle parameters of $\hat{G}(\tau\mid\theta)$ were:
shape ${\alpha = 1.813}$ and scale ${\beta = 2.199}$,
yielding $\hat{S}(\tau\mid\theta^*)$
that reasonably approximated the target $S(\tau)$.
\par
Comparing the estimated generation time to published serial interval distributions
(Figure~\ref{fig:distr-refs}) showed the following.
The mean generation time of $3.99$ was similar to
the mean serial interval of $3.96$
based on the negative-permitting distribution \cite{Du2020},
but shorter than mean serial interval based on non-negative distributions,
such as $5.12$ in \cite{Zhang2020} and $4.7$ in \cite{Nishiura2020}.
% SM: make it clear that using the same serial interval data for generation time
%     and the fitted serial interval distributions [perhaps in methods?]
% JK: not 100% sure what you mean here?
%     In estimating G(t) from S(t) and H(t), we always use
%     the same parametric definition of S(t) from Du2020, in eq (4).  %SM: ah got it. clear now with the word 'published' above. i missed that before.
The \sd of the generation time distribution was smaller at $2.96$
than the \sd of the negative permitting serial interval at $4.75$ \cite{Du2020},
but similar to the \sd of the non-negative serial interval distributions,
including $2.69$ in \cite{Zhang2020} and $2.9$ in \cite{Nishiura2020}.
% Finally, the distribution 95\% quantile was approximately $10$ days
% for the generation time distribution and
% non-negative serial interval distributions \cite{Zhang2020,Nishiura2020},
% but nearly $12$ days for the
% negative permitting serial interval distribution \cite{Du2020}.
% JK: This ^ has possible implications for quarantine,
%     but it feels out of scope, and I'd be happy to remove it
%     unless you feel it is important.  %SM: agree - does have improtant implications but does feel out of scope. agree with leaving it out.
\par
\begin{figure}
  \centering
  \includegraphics[width=\linewidth]{deconv}
  \caption{Recovered generation time distribution
    $\hat{G}(\tau\mid\theta^*)$
    based on \mle approximation of the serial interval distribution
    $S(\tau)$ by $\hat{S}(\tau\mid\theta^*)$
    and the incubation period distribution $H(\tau)$.}
  \label{fig:deconv}
\end{figure}
\par
Figure~\ref{fig:Re} shows
% SM: can we zoom-in to the later dates?
%     to show the differences more clearly after April 10 for example?
% JK: I've added in the appendix and referenced from here
%     but let me know if you think it should also appear in the main text.  %SM: looks good in appendix!
$\Re(t)$ for \covid in \gta, Canada
based on reported cases and estimated using
the generation time distribution versus
selected serial interval distributions reported in the literature.%
\footnote{Generation time and serial interval distributions
  are also illustrated in Figure~\ref{fig:distr-refs}.}
The $\Re(t)$ based on non-negative serial interval distribution was higher  %SM: but 'flipped' after R<1? explain and mention that?
versus $\Re(t)$ using the estimated generation time distribution
(e.g.\ using \cite{Zhang2020}:\
\rvalue{R-2020-03-16-S(t)-[NN]-Zhang-2020} vs
\rvalue{R-2020-03-16-G(t)-[this]} on March~16; and
\rvalue{R-2020-04-13-S(t)-[NN]-Zhang-2020} vs
\rvalue{R-2020-04-13-G(t)-[this]} on April~13).
% SM: values using an early date and a later date
% JK: I definitely agree the values are helpful, but not sure if we need 2 dates?
%     just don't want to overwhelm the big-picture trend with numbers...  %SM: agree but in reading it becomes more clear with the numbers you included as is (i.e. 2 dates).
%     Similarly, I originally had the 95% CI but it felt cluttered... %SM: agree with leaving the 95% CI out. Might be good to 'append' the time-series of outputs though as a long table as median and 95% credible intervals, what do you think?
% SM: currently the reason for this is not discussed.
%     Could put in discussion but I think would be good to explain the 'why' here.
% JK: How is this? :)  %SM: good!
Higher $\Re(t)$ can be attributed to
longer mean serial interval under non-negative distributions
versus the estimated mean generation time,
since inferred $\Re(t)$ must be higher
to compensate for longer delay between infections.
By contrast, the $\Re(t)$ estimated using
a negative-permitting serial interval distribution
was the smallest of all three approaches
(e.g.\ using \cite{Du2020}:\
\rvalue{R-2020-03-16-S(t)-[NP]-Du-2020} on March~16; and
\rvalue{R-2020-04-13-S(t)-[NP]-Du-2020} on April~13).
% SM: given reason/explain 'why' here.
% JK: is this sufficient? It could get very mathy to fully explain,
%     and Britton et al really do examine exactly this.  %SM: let the reader know that in Discussion (that explanation further detailed in...or as per..and summarized here)
In this case, lower $\Re(t)$ can be attributed to
increased variance in the negative-permitting serial interval distribution
versus in the generation time distribution,
as shown by \textcite{Britton2019}.
\par
\begin{figure}[h]
  \centering
  \includegraphics[width=\linewidth]{Re}
  \caption{$\Re(t)$ of \covid in \gta using
    serial interval versus generation time}
  \label{fig:Re}
  \floatfoot{Notation ---
    $S(\tau)$:~serial interval;
    $G(\tau)$:~generation time;
    [NP]:~negative-permitting;
    [NN]:~non-negative.
    See Figure~\ref{fig:Re-zoom}
    for zoom-in of later dates.
  }
\end{figure}