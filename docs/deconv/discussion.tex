We have demonstrated how to estimate
a non-negative generation time distribution
based on negative serial interval
and non-negative incubation period parametric distributions.
We showed that estimates of $\Re(t)$ will vary depending on
the approach used to approximate/estimate the generation time distribution.
Specifically, relative to the estimated generation time distribution,
non-negative and negative-permitting serial interval distributions
may over and underestimate $\Re(t)$, respectively.
\par
Our estimated generation time for \covid was
Gamma-distributed with mean 3.99 and \sd 2.96,
similar to results by \textcite{Ganyani2020} (Table~\ref{tab:distr}).
However in \cite{Ganyani2020}
person-level serial interval data on infector-infectee pairs
are required for a joint Bayesian model of
generation time, incubation period, and serial interval,
characterized via \textsc{mcmc} sampling.
In our approach, we used parametric distributions as inputs
because we did not have access to 
person-level and paired data.
As such, our approach could also use pooled estimates via meta-analyses of
serial interval and incubation period as parameter inputs.
\par
In several recent works \cite{You2020,Tang2020,Zhang2020a,Zhang2020},
non-negative serial interval distributions have been used
as an approximation of the generation time distribution
when estimating $\Re(t)$ for \covid.%
\footnote{In early characterizations of \covid
  \cite{Li2020,Zhang2020,Nishiura2020,Zhao2020,You2020},
  negative serial interval (and pre-symptomatic transmission)
  was either unobserved or considered implausible.}
We found that such an approximation may result in
overestimation of $\Re(t)$,
and thus overestimation of \covid transmission potential.
The finding that $\Re(t)$ may be overestimated
when using (non-negative) serial interval versus generation time
seemingly contradicts the conclusion of \textcite{Britton2019},
but can be explained as follows.
In our study of \covid,
the mean serial interval under non-negative distributions \cite{Zhang2020,Nishiura2020}
was longer than the mean generation time
estimated using negative-permitting serial interval \cite{Du2020}.
If time between infections is assumed to be longer,
then we would \emph{overestimate} $\Re(t)$ to fit the same incidence.
By contrast, \textcite{Britton2019} modelled
both serial interval and generation time as Gamma-distributed (non-negative),
resulting in equal means, but different variances.
They then showed how increased variance in the serial interval distribution
can cause \emph{underestimation} of $\Re(t)$, relative to the generation time distribution.
In fact, we also find underestimation of $\Re(t)$ when comparing
negative-permitting serial interval to the generation time distribution,
which had equal means.
\par
Our approach to estimating generation time had three notable limitations.
First, like similar works \cite{Kuk2005,Ganyani2020}, we assumed that
generation time and incubation period were independent,
although \textcite{Klinkenberg2011} showed that
correlation between the two exist in infections such as measles.
Second, as noted above, our approach did not use
person-level serial interval or incubation period data
such as in \cite{Ganyani2020} and \cite{Klinkenberg2011},
possibly resulting in compounding errors from
parametric approximation of both input (serial interval, incubation period)
and output (generation time) distributions.
However, depending on the availability and reliability of person-level data,
our approach could in some cases be favourable.
Finally, we did not perform uncertainty analysis using
the reported confidence intervals for
the serial interval and incubation period distribution parameters.
Future work could circumvent this limitation by exploring
joint estimation of the generation time, serial interval, and $\Re(t)$
within the same Bayesian framework as described by \cite{Cori2013}.