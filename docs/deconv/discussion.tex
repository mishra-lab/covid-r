We have demonstrated how to estimate
% SM: great, this is a nice key message / point #1
a non-negative generation time distribution
based on negative serial interval
and non-negative incubation period parametric distributions.
% SM: reads like we are the first to do this?
%     is the correct, or should we cite in introduction and methods other work?
% JK: AFAIK we are the first to do it with *parametric* inputs
%     and MLE using KL-divergence.
%     I have added to the next paragraph to hopefully clarify the differences.
We showed that estimates of $\Re(t)$ will vary depending on
the approach used to approximate/estimate the generation time distribution.
Specifically, relative to the estimated generation time distribution,
non-negative and negative-permitting serial interval distributions
may over and underestimate $\Re(t)$, respectively.
% SM: may have reversed the order!
%     [but I think key in summary paragraph is saying sometihng like this as key message / point #2]
% JK: agreed!
\par
Our estimated generation time for \covid was
Gamma-distributed with mean 3.99 and \sd 2.96,
similar to results by \textcite{Ganyani2020} (Table~\ref{tab:distr}).
However in \cite{Ganyani2020}
% as in \cite{Klinkenberg2011} for measles, % JK: feels out of place.
subject-level serial interval data on infector-infectee pairs
are required for a joint Bayesian model of
generation time, incubation period, and serial interval,
characterized via \textsc{mcmc} sampling.
In our approach, we used parametric distributions as inputs
because we did not have access to 
person-level and paired data. %SM: person not subject :) 
As such, our approach could also use pooled estimates via meta-analyses of
serial interval and incubation period as parameter inputs.
% JK: some things have moved. you previously wrote:
% SM: why not [how did they have line-listed data -- was it on serial interval]?
%     what data did we use? make it clear in intro / objectives
%     we are using serial interval data and incubation data from the literature
% JK: does this help clarify? SM: yes
\par
In several recent works \cite{You2020,Tang2020,Zhang2020a,Zhang2020},
% JK: Not sure if we do/not want to call out specific works;
%     on one hand we would like to show that it is a common problem,
%     but on the other, we don't want to be rude.
non-negative serial interval distributions have been used
as an approximation of the generation time distribution
when estimating $\Re(t)$ for \covid.%
\footnote{In early characterizations of \covid
  \cite{Li2020,Zhang2020,Nishiura2020,Zhao2020,You2020},
  negative serial interval (and pre-symptomatic transmission)
  was either unobserved or considered implausible.}
% JK: this too -- is it kind enough? %SM: yes, that's good. kind but objective.
We found that such an approximation may result in
overestimation of $\Re(t)$,
and thus overestimation of \covid transmission potential.
The finding that $\Re(t)$ may be overestimated
when using (non-negative) serial interval versus generation time
seemingly contradicts the conclusion of \textcite{Britton2019},
but can be explained as follows.
In our study of \covid,
the mean serial interval under non-negative distributions \cite{Zhang2020,Nishiura2020}
was longer than the mean generation time
estimated using negative-permitting serial interval \cite{Du2020}.
If time between infections is assumed to be longer,
then we would \emph{overestimate} $\Re(t)$ to fit the same incidence.
By contrast, \textcite{Britton2019} modelled
both serial interval and generation time as Gamma-distributed (non-negative),
resulting in equal means, but different variances.
They then showed how increased variance in the serial interval distribution
can cause \emph{underestimation} of $\Re(t)$, relative to the generation time distribution. %SM: nicely explained!
In fact, we also find underestimation of $\Re(t)$ when comparing
negative-permitting serial interval to the generation time distribution.
\par
Our approach to estimating generation time had three notable limitations.
First, like similar works \cite{Kuk2005,Ganyani2020}, we assumed that
generation time and incubation period were independent,
although \textcite{Klinkenberg2011} showed that
correlation between the two exist in infections such as measles.
% SM: did the paper actually show this, or did they cite somene else who showed it?
%     i.e. was it their own data that showed the correlation?
% JK: They actually showed it by estimating a correlation coefficient (Table 1)  %SM: great, have finally read the papers now :) 
%     based on old measles data (not sure if publicly available).
Second, as noted above, our approach did not use
person-level serial interval or incubation period data
such as in \cite{Ganyani2020} and \cite{Klinkenberg2011},
possibly resulting in compounding errors from
parametric approximation of both input (serial interval, incubation period)
and output (generation time) distributions.
% SM: I do not understand limtiation #2
%     (tried reading the draft a few times but not clear to me what is being said here
%     -- i think it is because we have not clearly stated what others have done
%     and what we are doing to add to that knowledge)
% JK: is this better? and hopefully clearer after including  %SM: yes, clear
%     the more detailed descriptions of Ganyani and Britton above?
However, depending on the availability and reliability of subject-level data,
our approach could in some cases be favourable.
Finally, we did not perform uncertainty analysis using
the reported confidence intervals for
the serial interval and incubation period distribution parameters.
Future work could circumvent this limitation by exploring
joint estimation of the generation time, serial interval, and $\Re(t)$
within the same Bayesian framework as described by \cite{Cori2013}.
% SM: the 'so what' is not clear (is missing).
%     I think some material in limitations are actually a bit like a 'how to use/how to do'?
%     will most people have line-listed data on serial interval?
%     in the case of this outbreak, that has not been the case
%     with challenges (limited) contact-tracing for example in the GTA.
%     What do the findings mean for the GTA re: under/over-estimation
%     (paragraph above re: implications).
% JK: is it better now? hopefully addressed the line-listed data issue.  %SM: yes
%     Re. implications, I think hopefully also clarified,
%     although I'm not sure about commenting on GTA specifically,  %SM: good point and makes sense. if reviewers ask, can always address at that stage.
%     since the results are not very dependent on I(t).