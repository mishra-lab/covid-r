We have demonstrated how to estimate
a non-negative generation time distribution
based on negative serial interval
and non-negative incubation period distributions.
Using reported parametric \covid distributions,
we estimated a Gamma-distributed generation time
with a mean of $4$ days and
$95\%$ of transmissions occurring before $10$ days.
When approximating the generation time distribution
by non-negative serial interval distribution,
$95\%$ of transmissions are still expected within $10$ days
\cite{Zhang2020,Nishiura2020},
but the proportion of pre-symptomatic infections may be underestimated
due to higher mean serial interval versus generation time
(Figure~\ref{fig:distr-refs}) \cite{Ganyani2020,Tindale2020}.
\par
Estimated effective reproduction number $\Re(t)$
based on the generation time distribution
was lower than $\Re(t)$ based on
non-negative serial interval distributions reported in the literature,
suggesting that the latter may overestimate
the infection transmission potential.
By contrast, estimation of $\Re(t)$ using
negative permitting serial intervals,
such as that reported by \textcite{Du2020},
may result in underestimated infection transmission potential.
\par
Our analysis has three major limitations.
First, we assumed that
generation time and incubation period were independent,
although \textcite{Klinkenberg2011} showed that
this is false at least in measles.
Second, we did not leverage line-listed (subject-level) data
to estimate the generation time distribution,
such as in \cite{Ganyani2020} and \cite{Klinkenberg2011}.
Rather, we focused on parametric distributions,
which could, for example, be obtained by meta-analysis.
Finally, we did not propagate uncertainty in the reported
serial interval and incubation period distributions
through our results to obtain confidence intervals for
the generation time parameterization.
To model distribution uncertainty,
future work could examine joint estimation of
the generation time, serial interval, and $\Re(t)$ within
the same Bayesian framework as described by \cite{Cori2013}.
