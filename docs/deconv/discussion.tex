%begin the discussion with 1-2 line summary paragraph [not showing new results]

We have demonstrated how to estimate  %SM: great, this is a nice key message / point #1
a non-negative generation time distribution
based on negative serial interval
and non-negative incubation period distributions. %reads like we are the first to do this? is the correct, or should we cite in introduction and methods other work?
We identified that $\Re(t)$ will vary depending on 
the approach used to either approximate the generation time 
or estimate generation time. Specifically, if an
estimated generation time is the benchmark, then
whether or not we fit the serial interval data to 
a negative-permitting or non-negative distribution will
over and underestimate $\Re(t)$ respectively.  %SM: may have reversed the order! [but I think key in summary paragraph is saying sometihng like this as key message / point #2]

\par %SM: in this paragraph, refer to Galyani and how the current study and published study estimates compare and how Ganyani used line-listed etc.?
Using reported parametric \covid distributions of incubation period and serial intervals,
we estimated a Gamma-distributed generation time
with a mean of $4$ days and                                     %belongs in results
$95\%$ of transmissions occurring before $10$ days. %belongs in results
When approximating the generation time distribution   %belongs in results? or does the 95% of transmission refer to prior literature? I found this sentence confusing and not sure how it fit into the story (especially in the summary paragraph of the discussion section)
by non-negative serial interval distribution,
$95\%$ of transmissions are still expected within $10$ days
\cite{Zhang2020,Nishiura2020},
but the proportion of pre-symptomatic infections may be underestimated  %underestimated using which method? what is being compared to what in this sentence?
due to higher mean serial interval versus generation time
(Figure~\ref{fig:distr-refs}) \cite{Ganyani2020,Tindale2020}.

\par %SM: reivse this paragraph in full - it is currently a repeat of the results. explain 'why' re: the difference [either here or in results, but at least in one place] and then implications.
Estimated effective reproduction number $\Re(t)$
based on the generation time distribution
was lower than $\Re(t)$ based on
non-negative serial interval distributions reported in the literature,  % already said in results. here - explain (a) why over/under estimate; and (b) the implications of over / under-estimating on tracking the epidemic and forecasting [which these models are also used to do].
suggesting that the latter may overestimate
the infection transmission potential.
By contrast, estimation of $\Re(t)$ using
negative permitting serial intervals,
such as that reported by \textcite{Du2020},
may result in underestimated infection transmission potential.

\par
Our analysis has three notable limitations.  %what does 'analyses' refer to here? the method to estimate generation time? the comparison of methods?
First, we assumed that
generation time and incubation period were independent,
although \textcite{Klinkenberg2011} showed that correlation between the two exist in infections such as measles.  %SM: did the paper actually show this, or did they cite somene else who showed it? i.e. was it their own data that showed the correlation?
Second, we did not leverage line-listed (subject-level) data  %I do not understand limtiation #2 (tried reading the draft a few times but not clear to me what is being said here -- i think it is because we have not clearly stated what others have done and what we are doing to add to that knowledge)
to estimate the generation time distribution,                          %the way things are written, we are forcing reviewer/reader to read the key papers we cite - i think have to make it more clear what the others did and exactly what it is we are doing in addition / what is our contribution - explicitly?
such as in \cite{Ganyani2020} and \cite{Klinkenberg2011}.   %why not [how did they have line-listed data -- was it on serial interval]? what data did we use? make it clear in intro / objectives we are using serial interval data and incubation data from the literature
Rather, we focused on parametric distributions,                    %why? what was the rationale?
which could, for example, be obtained by meta-analysis.     %not clear what is being said here ... meta-analyses of?
Finally, we did not propagate uncertainty in the reported
serial interval and incubation period distributions
through our results to obtain confidence intervals for
the generation time parameterization.                                 %clarify and cite. what is meant by propogation of uncertainty here?
To model distribution uncertainty,                                         %clarify - what is 'distribution uncertainty'? 
future work could examine joint estimation of
the generation time, serial interval, and $\Re(t)$ within
the same Bayesian framework as described by \cite{Cori2013}.


%SM: the 'so what' is not clear (is missing). i think some material in limitations are actually a bit like a 'how to use/how to do'? will most people have line-listed data on serial interval? in the case of this outbreak, that has not been the case with challenges (limited) contact-tracing  for example in the GTA. What do the findings mean for the GTA re: under/over-estimation (paragraph above re: implications).
