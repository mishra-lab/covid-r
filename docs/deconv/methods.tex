First, we designed a simple method to
% SM: excellent methods section .. nice flow and terrific work Jesse!
%     clearly you are thinking / thought of the same issues other experienced modelers did too :)
%     but did a very nice job of explaining it
% JK: Thanks! :D
recover the generation time distribution
from parametric definitions of
the incubation period and serial interval distributions,
similar to work by \textcite{Kuk2005} and \textcite{Britton2019}.
% SM: cite previous work here or better to include in discussion? what do you think?
% JK: I think good to acknowledge here (and not overstate our contributions)
%     but also discuss more in discussion.
Then, we applied this method to \covid using
published incubation period and serial interval distributions.
Finally, we compared $\Re(t)$ estimated for
the Greater Toronto Area (\gta) region of Canada
between March~08 and April~15, 2020 using
the estimated generation time distribution versus
negative permitting and non-negative serial interval distributions
reported in the literature.
% ==============================================================================
\subsection{Estimating the Generation Time Distribution}
Let $i$ and $i+1$ be the indices of an infector-infectee pair.
Let $f_i$ and $f_{i+1}$ be the respective times of infection,
such that $g_i = [f_{i+1} - f_i] \sim G(\tau)$ is the generation time.
Let $y_i$ be the time of symptom onset in case $i$,
such that $h_i = [y_i - f_i] \sim H(\tau)$ is the incubation period.
Finally, let $s_i = [y_{i+1} - y_i] \sim S(\tau)$ be the serial interval.
Figure~\ref{fig:nodes} illustrates the variables and distributions graphically.
We assume all distributions are independent,
although previous work has shown that
the generation time and incubation period may be correlated,
for example, in the context of measles \cite{Klinkenberg2011}.
\par
\begin{figure}
  \centering
  \newcommand{\gauss}[2]{1/(#2*sqrt(2*pi))*exp(-0.5*((\t-#1)/#2)^2)}
\definecolor{infector}{HTML}{cccccc}
\definecolor{infectee}{HTML}{0033ff}
\definecolor{symptoms}{HTML}{ff0000}
\begin{tikzpicture}
  \node (f1) [event=infector]             at ( 0\dx, 4\dx) {$f_i$};
  \node (f2) [event=infectee]             at ( 2\dx, 3\dx) {$f_{i+1}$};
  \node (y1) [event=infector!50!symptoms] at ( 0\dx, 2\dx) {$y_i$};
  \node (y2) [event=infectee!50!symptoms] at ( 2\dx, 1\dx) {$y_{i+1}$};
  \draw[arrow,->]   (f1) -- node[near start, above right]{$G(\tau)$} (f2);
  \draw[brace]      (y1) -- node[near end, below left]   {$S(\tau)$} (y2);
  \draw[arrow,->]   (f1) -- node[right]{$H(\tau)$} (y1);
  \draw[arrow,->]   (f2) -- node[right]{$H(\tau)$} (y2);
  \draw[arrow,->,dashed] (f2) -- ++( 2\dx,-1\dx);
  \draw[arrow,<-,dashed] (f1) -- ++(-2\dx, 1\dx);
  % distributions
  \node(t0) at (5\dx, 5.5\dx) {\rlap{$t$}};
  \node(t1) at (5\dx,-0.5\dx) {};
  \draw[plot={infectee}            {\t},domain=0:5] plot ({5\dx+0.5*\dx*\gauss{3}{.23}},{\t});
  \draw[plot={infector!50!symptoms}{\t},domain=0:5] plot ({5\dx+0.5*\dx*\gauss{2}{.28}},{\t});
  \draw[plot={infectee!50!symptoms}{\t},domain=0:5] plot ({5\dx+0.5*\dx*\gauss{1}{.36}},{\t});
  \draw[thick,arrow,->] (t0) -- (t1);
  \draw[thick] (4.8\dx,4\dx) -- (5.2\dx,4\dx) node{\rlap{$f_i$}};
\end{tikzpicture}
  \caption{Random variables involved in the serial interval}
  \label{fig:nodes}
  \floatfoot{Notation ---
    $i$: infector index;
    $i+1$: infectee index;
    $f_i$: time of infection;
    $y_i$: time of symptom onset;
    $G(\tau)$: generation time distribution;
    $H(\tau)$: incubation time distribution;
    $S(\tau)$: serial interval distribution.}
\end{figure}
\par
Rearranging, we have:
$s_i = g_i + h_{i+1} - h_i$.
The probability distribution of
the sum of independent random variables
is the convolution of their respective distributions~\cite{Hogg2005},
where convolution $\conv$ is defined as:
\begin{equation}\label{eq:conv}
  \left[H \conv F\right](\tau) = \int_{-\infty}^{+\infty} H(z) F(\tau - z) \,dz
\end{equation}
Thus $S(\tau)$ can be defined as:
\begin{equation}\label{eq:S}
  S(\tau) = G(\tau) \conv H(\tau) \conv H(-\tau)
\end{equation}
and $G(\tau)$ can be recovered using deconvolution $\deconv$ as:
\begin{equation}\label{eq:G}
  G(\tau) = \left[S(\tau) \deconv H(\tau)\right] \deconv H(-\tau)
\end{equation}
\par
Some definitions of $S(\tau)$ and $H(\tau)$
may yield forms of $G(\tau)$ via deconvolution in \eqref{eq:G}
which are implausible or intractable.
So, we defined a parametric form ${\hat{G}(\tau\mid\theta)}$,
and found parameters $\theta^*$ that
minimized the Kullback-Leibler divergence between
the observed $S(\tau)$ and ${\hat{S}(\tau\mid\theta)}$
obtained via $\eqref{eq:S}$ using ${\hat{G}(\tau\mid\theta^*)}$.
It can be shown that such parameters $\theta^*$
provide the maximum likelihood estimate (\mle) of
$S(\tau)$ under $\hat{G}(\tau\mid\theta)$.
In contrast to previous works \cite{Kuk2005,Britton2019,Ganyani2020},
our approach thus uses parametric definitions of $S(\tau)$ and $H(\tau)$,
versus individual-level serial interval and/or incubation period data.
% JK: not sure if this ^ belongs here...
% ==============================================================================
\subsection{Application}
% ------------------------------------------------------------------------------
\subsubsection{Generation Time}
We identified several parameterizations of
the \covid incubation period and serial interval
following the review by \textcite{Park2020} (Table~\ref{tab:distr}).%
\footnote{Some studies did not provide enough information
  to define a parametric form (e.g.\ only reported the mean).}
For our analysis, we used
the negative-permitting serial interval from \cite{Du2020} ({$N=468$}):
\begin{equation}
S(\tau) = \op{Norm}{\mu=3.96,~\sigma=4.75}
\end{equation}
and the incubation period from \cite{Lauer2020} ({$N=181$}):
\begin{equation}
H(\tau) = \op{Gam}{\alpha=5.807,~\beta=0.948}
\end{equation}
We assumed a Gamma parametric form for the generation time distribution
${\hat{G}(\tau\mid\theta)}$, with
$\theta = [\alpha\,(\textrm{shape}), \beta\,(\textrm{scale})]$,
for consistency with downstream assumptions used in calculating $\Re(t)$.
We then minimized the Kullback-Leibler divergence between
$S(\tau)$ and $\hat{S}(\tau\mid\theta)$ using
the Nelder-Mead optimization method
in the \mono{optimization} R package\,%
\footnote{\hreftt{https://cran.r-project.org/web/packages/optimization}}
to obtain the \mle generation time distribution parameters $\theta^*$.
% ------------------------------------------------------------------------------
\subsubsection{Effective Reproduction Number}
In the model described by \textcite{Cori2013},
the incidence $I$ at time $t$ is given by
the integral over all previous infections,
multiplied by their respective infectivity $\omega$
at time $\tau$ since infection,
collectively multiplied by the effective reproductive number
$\Re$ at time $t$.
This model yields the following definition of $\Re(t)$:
\begin{equation}
\Re(t) = I(t) {\left[\int_{0}^{t}{I(t-\tau) \omega(\tau) d\tau}\right]}^{-1}
\end{equation}
The infectivity profile $\omega(\tau)$ is equivalent to
the generation time distribution $G(\tau)$.
Given $I(t)$ and $G(\tau)$,
probabilistic estimates of $\Re(t)$
can then be resolved in a Bayesian framework,
as implemented in the \mono{EpiEstim} R package.%
\footnote{\hreftt{https://cran.r-project.org/web/packages/EpiEstim}}
\par
In order to quantify $\Re(t)$ of \covid in \gta, Canada,
we used reported cases in the region
with episode dates between March 09 and May 04 2020
as the incidence time series $I(t)$.%
\footnote{We did not stratify incidence by
  ``imported'' vs ``local'' transmission events,
  in order to quantify overall epidemic growth.
  The relative influence of different $G(\tau)$ approximations on $\Re(t)$
  should not be affected by this lack of stratification.
}
We smoothed $I(t)$ using a Gaussian kernel with $\sigma = 1$ day
to reflect uncertainty in reporting delay.
We then compared estimates of $\Re(t)$ using
% SM: Nice! would bring this up in objectives I think as well (see suggestions in intro section)
% JK: Done! Thanks :)
the \mle generation time distribution
versus serial interval distributions reported in the literature,
including negative-permitting (Normal \cite{Du2020}),
and non-negative (Gamma \cite{Zhang2020}, Log-Normal \cite{Nishiura2020})
distributions.