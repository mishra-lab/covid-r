\par\textbf{Background:}
The effective reproduction number $R_e(t)$ is the average number of new infections directly generated from each existing infection under the conditions at time $t$.
$R_e(t)$ can be calculated from the number of daily infections
and the time between subsequent infections---the infection-infection distribution, $G(t)$.
The infection-infection distribution $G(t)$ is often approximated by the symptom-symptom distribution $S(t)$,
because the time of infection can be difficult to determine.
However, if the time between infection and symptom onset---the infection-symptom distribution $H(t)$---varies substantially,
then the infectee can develop symptoms before the infector and $S(t)$ can be negative, such as in the case of \covid.
In this case, it may be improper to approximate $G(t)$ with $S(t)$.
\par\textbf{Methods:}
Given parametric equations for the symptom-symptom distribution $S(t)$ and the infection-symptom distribution $H(t)$,
we develop a method to recover the infection-infection distribution $G(t)$ using approximate deconvolution.
We then compare estimates of $R_e(t)$ for the Greater Toronto Area using $G(t)$ to those using $S(t)$;
two definitions of $S(t)$ are considered, which do and do not allow negative values, respectively.
\par\textbf{Results:}
We estimated the time between \covid infections $G(t)$ to be Gamma-distributed with mean 4.08 and standard deviation 3.19 days.
The negative-permitting distribution $S(t)$ had equal mean but larger variance than $G(t)$, resulting in underestimation of $R_e(t)$ relative to $G(t)$,
whereas the non-negative $S(t)$ had similar variance but larger mean, resulting in overestimation of $R_e(t)$.
\par\textbf{Discussion:}
Approximation of the infection-infection distribution $G(t) $with the symptom-symptom distribution $S(t)$
may result in biased estimates of the effective reproduction number $R(t)$.
The infection-infection distribution $G(t)$ can also be understood as the distribution of infectiousness;
thus accurately distinguishing $G(t)$ from $S(t)$ may also have implications for isolation interventions.
Future work should explore possible correlation between $S(t)$, $H(t)$, and $G(t)$ and
estimation of confidence intervals for distribution parameters.
\par\textbf{Open Science:}
The daily GTA case data used are not public.
All other analysis code and data are available at:\\\hreftt{https://github.com/mishra-lab/covid-r}